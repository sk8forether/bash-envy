\documentclass[a4paper,11pt]{article}
\usepackage[utf8]{inputenc}
\usepackage{geometry,amsmath,color,graphicx}
\geometry{margin=1in}


%opening
\title{\vspace{-10ex}AMOC at 34S}
\author{Tim Smith}
\date{\vspace{-3ex}}

%% --- New commands
\newcommand{\pderiv}[3][]{% \pderiv[<order>]{<func>}{<var>} 
  \ensuremath{\frac{\partial^{#1} {#2}}{\partial {#3}^{#1}}}}
  
\newcommand{\red}[1]{\textcolor{red}{#1}}
\newcommand{\degSym}{$^{\circ}$}

%% --- Figures
\graphicspath{ {../../samoc/figures/} }
\DeclareGraphicsExtensions{.png,.pdf}

\begin{document}

\maketitle

\begin{abstract}
  We are interested in the variability of AMOC at 34$^{\circ}$S over time scales ranging from a few months to ~10-20 years. This location is of particular interest because there are two pilot boundary arrays located here (at 34.5$^{\circ}$): on the Eastern boundary two sensors as part of the GoodHope program (Feb2008-Dec2010) and the Southwest Atlantic MOC array (deployed March 2009) on the Western boundary \cite{meinenSamoc}. These systems are made up of Inverted Echo Sounders (IES), with a pressure sensor (PIES), and in some places a current meter (CPIES). While these arrays were not put in place to monitor AMOC, we use the data provided to compare computer model calculations of the AMOC. Here we use the MITgcm parameters fit to observations as in ECCO v4 r2 (much more here) \cite{forgetEccov4}. Defining the AMOC as an objective function and using the adjoint mode of this code allows us to compute sensitivities of AMOC to various external forcings (wind stress, buoyancy fluxes, ...). The purpose here is to find the causal mechanisms of variability and the timescales on which they act. While a 20 year time frame is not nearly enough to establish statistically meaningful trends in AMOC's behavior, it is certainly true that continuing or reestablishing observation systems will give a clearer picture of AMOC's low frequency variability and trends. 
\end{abstract}

\section{Introduction} 
\label{intro}
	
	\begin{enumerate} 

	  \item AMOC variability is a widely studied topic and is generally agreed upon as a good metric for variability in global ocean circulation 
	  \item Understanding its fluctuations (e.g. as white noise or as deterministic trends) is essential in forecasting for example the transport of heat in a warming climate
	  \item \red{Obviously need to expand on this more, mention \cite{wunschAndHeimbach_AMOC} as well as studies where it is suggested that AMOC is declining in a realistic way due to climate change} 
	  \item There is an abundance of literature in the Northern hemisphere \red{largely due to number of observations??} 
	  \item this region is extremely important because of its connection to the Southern Ocean, which is an important region for global ocean circulation 
	  \item \red{Need to expand on why the southern ocean is important, mechanisms at play ... etc.}
	  \item There are sparse observations in the South Atlantic \red{is there a particular reason why this is the case?}
	  \item Most focus around AMOC is in the Northern hemisphere, here we discuss AMOC at the Southernmost tip of the Atlantic (34.5$^{\circ}$S)
	  \item We are interested in attributing variation in AMOC to external forcing fields such as wind stress, heat, and freshwater fluxes.
	  \item We do this by defining an objective function for AMOC at 34S (closest grid point on MITgcm to 34.5S) and compute sensitivities 
	  \item Use MITgcm, TAF, to compute sensitivities of objective function to external forcing fields
	  \item ECCO v4 r2 model set up and parameter values \red{Gael ftw}
	  \item Sensitivity experiments have been performed many many times for producing physically meaningful information into underlying important mechanisms (e.g. Patrick's timescales, Czeschel, Pillar, Marotzke, ... )
	  \item See robust behavior in sensitivity fields (compared to 25N) in wind stress especially 
	  \item Interestingly we do not see oscillatory behavior in the sensitivity patterns going backward in time as Czeschel does
	  \item We don't see similar physical patterns to \ref{pillar} in FWF
	  \item We don't need to compute 12 objective functions for reconstruction as in \ref{pillar}
 
	\end{enumerate} 

\section{Set up}
\label{setup}
  \subsection{Model Configuration}

	\begin{enumerate}
	  \item Here we study the sensitivity of AMOC over a twenty year period using the MIT general circulation model (MITgcm) in a global ocean configuration.
	  \item The grid used is llc90: roughly $1^{\circ} x 1^{\circ}$ with Lat/Lon configuration from 70$^{\circ}$S and 57$^{\circ}$N, with a cubed sphere configuration on either pole
	  \item 50 layers in the vertical 
	  \item ggl90 vertical mixing scheme \cite{ggl90}
	  \item Gent McWilliams parameterization of unresolved eddies (bolus velocity) \cite{GM}
	  \item Isopycnal diffusivity discretized via \cite{Redi}
	  \item Parameters take on values from 12 optimization cycles from ECCO framework \red{maximum a posteriori??}
	  \item External forcing from Era Interim .. files after smoothing and post processing in \cite{forgetECCOv4}. These can be downloaded from ecco-group.org.
	  \item more details in \cite{forgetECCOv4}
	\end{enumerate}

  \subsection{Sensitivities in an adjoint framework}

	\begin{enumerate}
	  \item An adjoint inverse modeling framework has been used in many previous studies to investigate the sensitivity of a physically meaningful quantity, evaluated as a cost or objective function \cite{marotzke} \cite{heimbach_timescales} \cite{pillar} \cite{czeschel}
	  \item Adjoint (reverse) mode of the model is computed via the automatic differentiation package TAF Giering 2005 (see forget), FastOpt, heimbach hill and Giering 2005 (efficient exact adjoint )

	  \item Here we define the objective function as AMOC at 34$^{\circ}$S as in Eqn. \ref{amoc}. This is the closest latitude band resolved in the model configuration to 34.5$^{\circ}$S: the most Southern latitude band in the Atlantic that touches both African and South American continents. 

	  \item We compute sensitivities using the adjoint mode of the MITgcm, based on the same framework as in the ECCO version 4 project (where here sensitivities are computed with respect to a model-data misfit objective function). 
	  \item Using the generalized framework for establishing controls and objective functions as outlined in \cite{forgetECCOv4} makes this process highly streamlined.
	  \item We are interested in the sensitivity of amoc to external forcing. The framework described above allows us to generally define the controls and cost function as such, and compute sensitivities directly through the adjoint framework.
	  \item this builds the gradient dJ/du, which is computed much tractably than if we perturbed each external forcing control at each point in the globe to find its response in dJ after 20 years of integration ... 
	\end{enumerate}

  \subsection{AMOC definition}
  
    The domain of interest is the zonal cross section of the Atlantic ocean at 34$^{\circ}$S, from the Western coast in the Aghulas Basin to the Cape of Good Hope Basin in the East. In cartesian coordinates, we denote this as: $x \in [x_W,x_E], y = y(\phi=34^{\circ}S), z \in [-H,\eta(y,t)]$ (spherical coordinates are used in the computation, however).  We define the meridional overturning stream function as follows: 
    
    \begin{equation}
      \pderiv[]{\psi_{MOC}}{y}(y,z,t) = \int_{x_W}^{x_E}w(x,y,z,t)\,dx \qquad\text{;}\qquad \pderiv[]{\psi_{MOC}}{z}(y,z,t) = -\int_{x_W}^{x_E}v(x,y,z,t)\,dx .
     \label{eq:mocStf}
    \end{equation}

   Assuming no normal flow at $z = -H$, the stream function is given by:
   \begin{equation}
    \psi_{MOC}(y,z,t) = -\int_{-H}^{z}\int_{x_W}^{x_E}v(x,y,z,t)\,dx\,dz .
    \label{eq:mocStf2}
   \end{equation}

   Note that with this definition, volume transport moving northward is positive. AMOC is defined as the maximum value of this stream function: 
   
   \begin{equation}
    AMOC(y,z_{max},t) = \max_{-H < z < \eta(y,t)}{\psi_{MOC}(y,z,t)} ,
    \label{eq:amoc}
   \end{equation}

   where it is important to note that this value is a function of time and latitude (here denoted as $y$). Here we use monthly mean values for meridional velocity ($\bar{v}$), so that AMOC as a monthly mean quantity. Using results from ECCOv4r2 \cite{forgetEccov4}, the overturning stream function along with its mean and standard deviation over the period 1992-2011 is shown in Fig. \ref{fig:mocStfStats}. At this location and over this time frame, AMOC has a mean value of 14.25 Sv with a standard deviation of 3.14 Sv, where the maximizing depth has a mean and standard deviation 1422m +/- 56m (median depth 1461.4m). Note that the vertical grid spacing at this depth is approximately 100m. \red{Mention that AMOC is normally distributed, cite \cite{wunschAndHeimbach_AMOC}? give measure of normality?}. These are shown in Fig. \ref{fig:samocStats}. Compared to calculations by Meinen et al \cite{meinenSamoc}, these values are slightly lower in the mean, and show far less variation on monthly time periods: they show, in some months, variation up to 30 Sv. I'll discuss these details in section \ref{obsSAMOC}. 

   
   \begin{figure}
    \centering
    \includegraphics[width=\textwidth]{mocStfStats}
    \caption{Overturning stream function at 34\degSym S computed using monthly mean velocity fields from ECCOv4r2.}
    \label{fig:mocStfStats}
   \end{figure}
   
   \begin{figure}
    \centering
    \includegraphics[width=\textwidth]{samocStats}
    \caption{(Top) AMOC computed at 34\degSym S, using monthly mean velocity fields from ECCOv4r2. (Bottom) The maximizing depth at which AMOC is computed during that month. The mean value and +/-1$\sigma$ values are given by the solid gray and dashed gray lines, respectively.}
    \label{fig:samocStats}
   \end{figure}


  

   \subsection{Observers SAMOC}
    \label{obsSAMOC}
    
    Following e.g. Meinen et al \cite{meinenSamoc} and McCarthy \cite{mcCarthy}, we construct the observers AMOC at 34\degSym S with the following components: 
    
    \begin{enumerate}
     \item Ekman transport: Computed using monthly mean of 6 hour ERA-Interim re-analysis fields (Dee et al. 2011), interpolated to ocean grid at surface over 1992-2011. The volumetric transport at latitude $y$ is: %
     
     \begin{equation}
      T_{ek}(y,t) = -\int_W^E\frac{\tau_x}{f\rho} \, dx
      \label{eq:ekTransp}
     \end{equation}
     
     where $f$ is the Coriolis paramter at 34\degSym S, $\rho = \rho_0 = 1029 kg/m^3$. 
     
     \item Geostrophic transport: Computed based on the geopotential height anomaly:
     
     \begin{equation}
      T_{geo}(y,z,t) = \int_W^E(v-v_r)dx = \dfrac{1}{f}\int_W^E \dfrac{dp}{\rho} = - \dfrac{1}{f} \int g\,dz = \dfrac{1}{f} \int d\phi= \dfrac{1}{f} (\phi'_E(z) - \phi'_W(z))
      \label{eq:geoTransp}
     \end{equation}
     
     where $\phi'$ is the geopotential surface anomaly which is made up from the surface and hydrostatic components:
     
     \begin{equation}
      \phi'_s \simeq \eta b_0 = \eta g \qquad\text{;}\qquad \phi'_{hyd} = \int_z^{z_{surf}} (b-b_0)dz .
      \label{eq:phiEqns}
     \end{equation}
    
    \item Reference transport: since the geostrophic component is computed relative to a certain level of flow, this is required. Here I use bottom pressure to compute flow at the ocean floor for the reference flow: 
    
    \begin{equation}
     T_{ref}(y,t) = \dfrac{1}{f} \int d\phi= \dfrac{1}{f} (\phi'_E(z) - \phi'_W(z))
     \label{eq:refTransp}
    \end{equation}

    where here bottom pressure anomaly divided by density is used for $\phi'$. 
 
    \end{enumerate}
    
    

 \section{Sensitivity Maps}

  Here I want to describe the sensitivity maps of the cost function, $\mathcal{J} = AMOC$, to various forcings $\pderiv{\mathcal{J}}{F}$. 
  Also I want to talk about the differences between what is seen in Pillar 
  Similarities in immediate response between all 4 sensitivities 
  

   \subsection{34S Plots}
   \subsubsection{Zonal Wind Stress}
	Note how this is essentially the reflection of sensitivity patterns at 25N. 
    \begin{enumerate}
	\item The original symmetric band around the cost function latitude is now positive, since an increase in zonal winds will cause net transport ``to the left'' due to the Coriolis effect. 
	\item The positive anomaly on top of the latitude band (due to baroclinic Rossby waves steering pressure perturbations around the Mid Atlantic Ridge) is of the opposite sign as before. Also, a fainter signal. 
	\item The Eastern boundary current carries a negative anomaly to the gulf of Guinea, which then travels rapidly along the equatorial wave guide (Kelvin dual waves). the exact opposite sign as in Northern hemisphere.
	\item In its wake a positive anomaly forms, which travels Westward \red{still unsure of its origin, and why this turns into an oscillating wave structure traveling Westward across the equator}. 
	\item \red{Baroclinic?} Rossby waves then carry these signals Westward, tilting from SW to NE due to phase speed as mentioned.
	\item Anomalies which travel Southward die out rapidly, and it seems like they get carried by Rossby waves, into the ACC, and pushed into the rough bathymetry of the Drake Passage. At this point the signal has died out. 
	\item \red{Fast barotropic waves carry signal rapidly Northward up the Western coast of SA (an Eastern boundary for the Pacific). Why does this signal oscillate up the coast?} 
    \end{enumerate}
   \subsubsection{Meridional Wind Stress}
    \begin{enumerate}
	\item Can see Mid Atlantic Ridge with + to right, and - to left. 
	\item \red{Is the dipole on the Western side, just a bit off the coast due to rough topography there? This is at the North of the Argentine Basin where the Rio Grande Rise is located.}
	\item Also on the Western side is the Wlave ridge surrounding the Cape Basin
	\item Don't see any other processes communicating these anomalies between coasts other than Rossby waves. 
	\item \red{baroclinically unstable flanks along subtropical gyres?}
	\item See damping to viscous dissipation at high latitudes with low rossby wave speeds
    \end{enumerate} 
   \subsubsection{Heat Flux} 
    \begin{enumerate} 
	\item Note that this does have a seasonal signal 
	\item Note how there is no oscillation and it doesn't matter how far back in time we go after the readjustment period, the signatures look the same with a different magnitude
	\item Important regions ... where is the memory? it's where the water needs to sink ... 
	\item note that Greenland area does not incorporate any ice melt processes 
    \end{enumerate} 
   \subsubsection{FWF}
    \begin{enumerate}
	\item Note that there is no seasonal cycle
	\item Note that after a readjustment period there is a ''breathing`` in the important regions but for the most part things stay the same 
	\item again, no oscillation over long time period
    \end{enumerate} 


 \section{Sensitivity means}

   \begin{figure}
    \centering
    \includegraphics[width=\textwidth]{nov.240mo/adjMean_hflux}
    \caption{Mean sensitivity at 34S to net heat flux downward ( $ >0 \implies $ ocean heating induces positive AMOC anomaly). November cost function. }
    \label{fig:novhflux}
   \end{figure}

   \begin{figure}
    \centering
    \includegraphics[width=\textwidth]{nov.240mo/adjMean_sflux}
    \caption{Mean sensitivity at 34S to net fresh water flux ( $>0 \implies$ net freshening induces positive AMOC anomaly). November cost function.}
    \label{fig:novsflux}
   \end{figure}

  \subsection{Linear Sensitivity}
  \label{linearSensitivity}
  
  We define an objective function $\mathcal{J}$ as the monthly mean value of AMOC as in Eqn. \ref{eq:amoc}. Using the adjoint mode of the MITgcm, we compute the sensitivity of the objective function to external forcing: $\pderiv{\mathcal{J}}{\mathcal{F}}$. In this case, $\mathcal{F}$ consists of the near surface air temperature, humidity, wind stress, precipitation, and radiative forcing.  
	
  We take the objective function to be split into its 20 year mean and monthly anomaly components: 
	\begin{equation}
	  \mathcal{J}_i(t) = \bar{\mathcal{J}_i}(t) + \mathcal{J}_i'(t) \qquad i \in [1, 2, ... 12]
	\end{equation}

  Here the subscript $i$ denotes the month in which the cost function is evaluated (January, February, etc). We hypothesize that the monthly anomaly in AMOC, $\mathcal{J}_i'(t)$, is linearly dependent on external forcing such that the AMOC anomaly for a given month, $i$, can be reconstructed offline via the convolution integral:
 
	\begin{equation}
	  \mathcal{J}_i'(t) = \int_{1992}^{t}\int_x \int_y\pderiv{\mathcal{J}_i}{\mathcal{F}}(x,y,\tau-t)\mathcal{F}'(x,y,\tau)\, dxdyd\tau  
	  \label{eq:reconstruct}
	\end{equation}

  \subsubsection{Reconstructions}

  reconstruction of AMOC at 34S. ain't it pretty.

   \begin{figure}
    \centering
    \includegraphics[width=\textwidth]{reconstruct.34S.exf_orig/full_reconstruct_mean}
    \caption{SAMOC reconstructed with monthly accumulated sensitivities and forcing files (wind stress, air temperature, humidity, lw/sw down, precip. ``Memory'' time is 20 years.}
    \label{fig:fullReconstruction}
   \end{figure}

  
\begin{thebibliography}{3}

  \bibitem{meinenSamoc}
  Meinen, et al. MOC Variability at 34.5S...
  
  \bibitem{forgetECCOv4}
  Forget, Gael, et al. \textit{ECCO version 4: an integrated framework for non-linear inverse modeling and global ocean state estimation.} Geoscientific Model Development 8 (2015): 3071-3104.
  
  \bibitem{wunschLinear}
  Carl Wunsch, ``Covariances and linear predictability of the Atlantic Ocean'', Deep Sea Research Part II: Topical Studies in Oceanography, Volume 85, January 2013, Pages 228-243, ISSN 0967-0645, http://dx.doi.org/10.1016/j.dsr2.2012.07.015.
  
  \bibitem{wunschAndHeimbach_AMOC}
  Wunsch, Carl, and Patrick Heimbach. ``Two decades of the Atlantic meridional overturning circulation: Anatomy, variations, extremes, prediction, and overcoming its limitations." Journal of Climate 26.18 (2013): 7167-7186.
  
  \bibitem{lozier}
  Lozier, ``Overturning assumptions.'' 

  \bibitem{mcCarthy}
  McCarthy, G. D., et al. ``Measuring the Atlantic meridional overturning circulation at 26 N." Progress in Oceanography 130 (2015): 91-111.

  \bibitem{heimbach_timescales}
  Heimbach, P., et al. ``Timescales and regions of sensitivity ...''

  \bibitem{pillar}
  Pillar et al ``Dynamical Attribution of Recent Variability in Atlantic Overturning''

  \bibitem{czeschel}
  Czeschel et al: oscillatory sensitivity 

  \bibitem{redi}
  Redi, 1982

  \bibitem{ggl90}
  Gaspar et al., 1990

  \bibitem{gm}
  Gent and McWilliams, 1990

  \bibitem{marotzke}
  Marotzke's paper on sensitivity for physically meaningful objective function

\end{thebibliography}

\end{document}
