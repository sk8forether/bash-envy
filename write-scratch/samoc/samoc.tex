\documentclass[a4paper,11pt]{article}
\usepackage[utf8]{inputenc}
\usepackage{geometry,amsmath,color,graphicx}
\geometry{margin=1in}


%opening
\title{\vspace{-10ex}AMOC at 34S}
\author{Tim Smith}
\date{\vspace{-3ex}}

%% --- New commands
\newcommand{\pderiv}[3][]{% \pderiv[<order>]{<func>}{<var>} 
  \ensuremath{\frac{\partial^{#1} {#2}}{\partial {#3}^{#1}}}}
  
\newcommand{\red}[1]{\textcolor{red}{#1}}
\newcommand{\degSym}{$^{\circ}$}

%% --- Figures
\graphicspath{ {../../samoc/figures/} }
\DeclareGraphicsExtensions{.png,.pdf}

\begin{document}

\maketitle

\begin{abstract}
  
  We are interested in the variability of AMOC at 34$^{\circ}$S over time scales ranging from a few months to ~10-20 years. This location is of particular interest because there are two pilot boundary arrays located here (at 34.5$^{\circ}$): on the Eastern boundary two sensors as part of the GoodHope program (Feb2008-Dec2010) and the Southwest Atlantic MOC array (deployed March 2009) on the Western boundary \cite{meinenSamoc}. These systems are made up of Inverted Echo Sounders (IES), with a pressure sensor (PIES), and in some places a current meter (CPIES). While these arrays were not put in place to monitor AMOC, we use the data provided to compare computer model calculations of the AMOC. Here we use the MITgcm parameters fit to observations as in ECCO v4 r2 (much more here) \cite{forgetEccov4}. Defining the AMOC as an objective function and using the adjoint mode of this code allows us to compute sensitivities of AMOC to various external forcings (wind stress, buoyancy fluxes, ...). The purpose here is to find the causal mechanisms of variability and the timescales on which they act. While a 20 year time frame is not nearly enough to establish statistically meaningful trends in AMOC's behavior, it is certainly true that continuing or reestablishing observation systems will give a clearer picture of AMOC's low frequency variability and trends. 
\end{abstract}

\section{Introduction} 
\label{intro}
	
	\begin{enumerate} 

	  \item AMOC variability is a widely studied topic and is generally agreed upon as a good metric for variability in global ocean circulation 
	  \item Understanding its fluctuations (e.g. as white noise or as deterministic trends) is essential in forecasting for example the transport of heat in a warming climate
	  \item \red{Obviously need to expand on this more, mention \cite{wunschAndHeimbach_AMOC} as well as studies where it is suggested that AMOC is declining in a realistic way due to climate change} 
	  \item this region is extremely important because of its connection to the Southern Ocean, which is an important region for global ocean circulation 
	  \item \red{Need to expand on why the southern ocean is important, mechanisms at play ... etc.}
	  \item There are sparse observations in the South Atlantic \red{is there a particular reason why this is the case?}
	  \item Most focus around AMOC is in the Northern hemisphere, here we discuss AMOC at the Southernmost tip of the Atlantic (34.5$^{\circ}$S)
	  \item We are interested in attributing variation in AMOC to external forcing fields such as wind stress, heat, and freshwater fluxes.
	  \item We do this by defining an objective function for AMOC at 34S (closest grid point on MITgcm to 34.5S) and compute sensitivities 
	  \item Use MITgcm, TAF, to compute sensitivities of objective function to external forcing fields
	  \item ECCO v4 r2 model set up and parameter values \red{Gael ftw}
	  \item Sensitivity experiments have been performed many many times for producing physically meaningful information into underlying important mechanisms (e.g. Patrick's timescales, Czeschel, Pillar, Marotzke, ... )
	  \item See robust behavior in sensitivity fields (compared to 25N) in wind stress especially 
	  \item Interestingly we do not see oscillatory behavior in the sensitivity patterns going backward in time as Czeschel does
	  \item We don't see similar physical patterns to \ref{pillar} in FWF
	  \item We don't need to compute 12 objective functions for reconstruction as in \ref{pillar}
 
	\end{enumerate} 

\section{Set up}
\label{setup}
  \subsection{Model Configuration}

	\begin{enumerate}
	  \item Here we study the sensitivity of AMOC over a twenty year period using the MIT general circulation model (MITgcm) in a global ocean configuration.
	  \item The grid used is llc90: roughly $1^{\circ} x 1^{\circ}$ with Lat/Lon configuration from 70$^{\circ}$S and 57$^{\circ}$N, with a cubed sphere configuration on either pole
	  \item 50 layers in the vertical 
	  \item ggl90 vertical mixing scheme \cite{ggl90}
	  \item Gent McWilliams parameterization of unresolved eddies (bolus velocity) \cite{GM}
	  \item Isopycnal diffusivity discretized via \cite{Redi}
	  \item Parameters take on values from 12 optimization cycles from ECCO framework \red{maximum a posteriori??}
	  \item External forcing from Era Interim .. files after smoothing and post processing in \cite{forgetECCOv4}. These can be downloaded from ecco-group.org.
	  \item External forcing describes the atmospheric state: air temperature at 2 meters above sea level, radiative surface fluxes (longwave and shortwave), humidity, precipitation, and wind stress. Runoff is input as a binary file \red{climatological mean?} 
	  \item No Newtonian relaxation terms are used except in the parameterization of sea ice. Described in \red{Duffy 1999 \& Nguyen 2009}.
	  \item From the atmospheric state inputs, the net heat and salt fluxes (including evaporation) are computed through the Bulk Formula \cite{yaeger2004}
	  \item more details in \cite{forgetECCOv4}
	\end{enumerate}

  \subsection{Sensitivities in an adjoint framework}

	\begin{enumerate}
	  \item An adjoint inverse modeling framework has been used in many previous studies to investigate the sensitivity of a physically meaningful quantity, evaluated as a cost or objective function \cite{marotzke} \cite{heimbach_timescales} \cite{pillar} \cite{czeschel}
	  \item Adjoint (reverse) mode of the model is computed via the automatic differentiation package TAF Giering 2005 (see forget), FastOpt, heimbach hill and Giering 2005 (efficient exact adjoint )
	  \item Some processes are approximated (e.g. Arctic ice melt, ggl90? )

	  \item Here we define the objective function as AMOC at 34$^{\circ}$S as in Eqn. \ref{amoc}. This is the closest latitude band resolved in the model configuration to 34.5$^{\circ}$S: the most Southern latitude band in the Atlantic that touches both African and South American continents. 

	  \item We compute sensitivities using the adjoint mode of the MITgcm, based on the same framework as in the ECCO version 4 project (where here sensitivities are computed with respect to a model-data misfit objective function). 
	  \item Using the generalized framework for establishing controls and objective functions as outlined in \cite{forgetECCOv4} makes this process highly streamlined.
	  \item We are interested in the sensitivity of amoc to external forcing. The framework described above allows us to generally define the controls and cost function as such, and compute sensitivities directly through the adjoint framework.
	  \item this builds the gradient dJ/du, which is computed much tractably than if we perturbed each external forcing control at each point in the globe to find its response in dJ after 20 years of integration ... 
	\end{enumerate}

  \subsection{AMOC definition}
  
    The domain of interest is the zonal cross section of the Atlantic ocean at 34$^{\circ}$S, from the Western coast in the Aghulas Basin to the Cape of Good Hope Basin in the East. In cartesian coordinates, we denote this as: $x \in [x_W,x_E], y = y(\phi=34^{\circ}S), z \in [-H,\eta(y,t)]$ (spherical coordinates are used in the computation, however).  We define the meridional overturning stream function as follows: 
    
    \begin{equation}
      \pderiv[]{\psi_{MOC}}{y}(y,z,t) = \int_{x_W}^{x_E}w(x,y,z,t)\,dx \qquad\text{;}\qquad \pderiv[]{\psi_{MOC}}{z}(y,z,t) = -\int_{x_W}^{x_E}v(x,y,z,t)\,dx .
     \label{eq:mocStf}
    \end{equation}

   Assuming no normal flow at $z = -H$, the stream function is given by:
   \begin{equation}
    \psi_{MOC}(y,z,t) = -\int_{-H}^{z}\int_{x_W}^{x_E}v(x,y,z,t)\,dx\,dz .
    \label{eq:mocStf2}
   \end{equation}

   Note that with this definition, volume transport moving northward is positive. AMOC is defined as the maximum value of this stream function: 
   
   \begin{equation}
    AMOC(y,z_{max},t) = \max_{-H < z < \eta(y,t)}{\psi_{MOC}(y,z,t)} ,
    \label{eq:amoc}
   \end{equation}

   where it is important to note that this value is a function of time and latitude (here denoted as $y$). Here we use monthly mean values for meridional velocity ($\bar{v}$), so that AMOC as a monthly mean quantity. Using results from ECCOv4r2 \cite{forgetEccov4}, the overturning stream function along with its mean and standard deviation over the period 1992-2011 is shown in Fig. \ref{fig:mocStfStats}. At this location and over this time frame, AMOC has a mean value of 14.25 Sv with a standard deviation of 3.14 Sv, where the maximizing depth has a mean and standard deviation 1422m +/- 56m (median depth 1461.4m). Note that the vertical grid spacing at this depth is approximately 100m. \red{Mention that AMOC is normally distributed, cite \cite{wunschAndHeimbach_AMOC}? give measure of normality?}. These are shown in Fig. \ref{fig:samocStats}. Compared to calculations by Meinen et al \cite{meinenSamoc}, these values are slightly lower in the mean, and show far less variation on monthly time periods: they show, in some months, variation up to 30 Sv. I'll discuss these details in section \ref{obsSAMOC}. 

   
   \begin{figure}
    \centering
    \includegraphics[width=.7\textwidth]{mocStfStats}
    \caption{Overturning stream function at 34\degSym S computed using monthly mean velocity fields from ECCOv4r2.}
    \label{fig:mocStfStats}
   \end{figure}
   
   \begin{figure}
    \centering
    \includegraphics[width=.7\textwidth]{samocStats}
    \caption{(Top) AMOC computed at 34\degSym S, using monthly mean velocity fields from ECCOv4r2. (Bottom) The maximizing depth at which AMOC is computed during that month. The mean value and +/-1$\sigma$ values are given by the solid gray and dashed gray lines, respectively.}
    \label{fig:samocStats}
   \end{figure}


 \section{Sensitivity Maps}

  
  Here I want to describe the sensitivity maps of the cost function, $\mathcal{J} = AMOC$, to various forcings $\pderiv{\mathcal{J}}{F}$. 
  Also I want to talk about the differences between what is seen in Pillar 
  Similarities in immediate response between all 4 sensitivities 
  

   \subsection{Zonal Wind Stress}
    \begin{enumerate}
	\item In the month prior to evaluating the cost function, there is a positive band around 34$^{\circ}$S, indicating that an increase in zonal winds will increase AMOC strength. This positive sensitivity band indicates that a stronger zonal wind will increase Ekman transport locally, which is ``to the left'' in the Southern hemisphere. 
	\item The sensitivity signals are much stronger across the Pacific and Indian ocean basins when compared to AMOC sensitivities measured around 26$^{\circ}$N, (compare to e.g. \cite{pillar} \cite{heimbach_timescales}). \red{... how to interpret these regions? unsure of the processes which are responsible for transporting these sensitivities.}
	\item Note that the sensitivities in shallow water regions (e.g. near the Bering Strait, North of Australia) are left over from the initial barotropic response. In reverse, a barotropic wave propagates from the objective function evaluation through the global ocean in roughly a day. However, since the barotropic wave speed is $\sqrt{gH}$, $H$ the water depth, these waves propagate slowly in these regions and linger for roughly a week or two. 
	\item \red{Fast barotropic waves carry signal rapidly Northward up the Western coast of SA (an Eastern boundary for the Pacific). Why does this signal oscillate up the coast?} 
	\item \red{Dipole around mid atlantic ridge showing the steering of pressure perturbations around the mid atlantic ridge. How to know this for sure? Access to Pillar's dissertation?}
	\item From a lead time of 1 to 2 months, the Eastern boundary current carries a negative anomaly to the gulf of Guinea, which then travels rapidly along the equatorial wave guide as a Kelvin dual wave. This signal is exactly as seen in e.g. \cite{pillar}, highlighting the importance of these linear wave dynamics as a sensitivity pathway throughout the Atlantic basin.
	\item As the reverse Kelvin waves travel poleward along the coast, a positive anomaly forms in its wake, which travels Eastward via (dual) baroclinic Rossby waves. These patterns are easily detected by their tilting from the Southwest to Northeast, due to the wave speed which is proportional to $\beta$. \red{could this be due to Ekman upwelling? why does this turn into an oscillating wave structure traveling Westward across the equator}.
	\item Similarly, these baroclinic Rossby waves can be seen near the equator of the Pacific and Indian ocean basins. The signals in the Pacific die out beyond roughly $\pm10^{\circ}$ beyond the equator, again indicating the importance of the equatorial region as a key pathway for communicating anomalies on monthly timescales. The baroclinic Rossby waves in the Indian ocean are much more apparent with higher frequency output (not shown).
	\item Anomalies which travel Southward dissipate rapidly as they are carried by dual Rossby waves into the ACC and pushed into the rough bathymetry of the Drake Passage. 
	\item \red{Interestingly, the sensitivities in zonal wind stress do not go farther south than $\sim$36$^{\circ}$S, which should be the mechanism for wind driven upwelling. Kuhlbrodt mentions that in the Southern ocean wind driven upwelling is an important mechanism for AMOC circulation, but in terms of measuring this we can't see any contribution from this component. Could be due to the strength of the ACC in this region. Beyond one month don't see any sensitivities in the Southern Ocean. Also could just be that 34S is too far away from this upwelling to be important.}
    \end{enumerate}
   \subsection{Meridional Wind Stress}
    \begin{enumerate}
	\item At a lead time of one month, we can see the Mid Atlantic Ridge in between a positive sensitivity to the East and negative to the West. 
	\item \red{Is the dipole on the Western side, just a bit off the coast due to rough topography there? This is at the North of the Argentine Basin where the Rio Grande Rise is located. This could be similar to the pressure steering mentioned in Pillar, except from all four sides: the rise seems to be around the center of 2 dipoles at one month lead time. }
	\item \red{Could the dipole to the East of the Mid Atlantic ridge be caused by the Walvis ridge? The line seems to follow Northeast just along where the ridge would be. Negative sensitivities are just below the ridge, where an increase in meridional wind stress can cause the waves to reflect off the ridge (depth $<1000m$? source?). The negative sensitivity just to the left gets advected in reverse up the Benguela current, then dissipates as it gets pushed into the African continent.}
	\item \red{What does the positive strip along the African coast represent? Does this increase the strength of the Benguela current, and potentially cause Ekman upwelling which feeds into the South Equatorial Current?}
	\item \red{Why would an increase in wind stress along the Agulhas current increase AMOC?}
	\item Can see residual from barotropic response in shallow waters around the globe. 
	\item After one month, baroclinic Rossby waves can be seen tilting the sensitivity patterns to the Northeast. 
	\item \red{baroclinically unstable flanks along subtropical gyres?}
	\item See damping to viscous dissipation at high latitudes with low rossby wave speeds
    \end{enumerate} 
   \subsection{Heat Flux} 
    \begin{enumerate} 
	\item In the first month, sensitivity patterns cover the entire globe as part of a fast barotropic readjustment period. Note that this is not exhibited by Pillar or Czeschel \cite{pillar}, \cite{czeschel} since the model configuration used does not include volume changes to thermal expansion. 
	\item Note that this does have a seasonal signal 
	\item Note how there is no oscillation and it doesn't matter how far back in time we go after the readjustment period, the signatures look the same with a different magnitude
	\item Important regions ... where is the memory? it's where the water needs to sink ... 
	\item note that Greenland area does not incorporate any ice melt processes 
	\item After the initial readjustment period:
	\item A seasonal cycle develops, going backward in time. 
	\item Main difference here is the connection with the global ocean: the Pacific and Indian oceans remain strongly sensitive regions over decadal timescales because of the connection to the Southern Ocean at 34S. Even though these patterns are not particularly strong, it takes roughly 10 years or more for the basin-wide signals to decay by an order of magnitude. (when considering the spatial RMS of sensitivity as a function of time).  
	\item The strong ACC overpowers the reverse Rossby waves, and sensitivities are carried Westward. The signal dissipates 
    \end{enumerate} 
   \subsection{FWF}
    \begin{enumerate}
	\item Note that there is no seasonal cycle
	\item Note that after a readjustment period there is a ''breathing`` in the important regions but for the most part things stay the same 
	\item again, no oscillation over long time period
    \end{enumerate} 


  \section{Linear Sensitivity}
  \label{linearSensitivity}
  
  We define an objective function $\mathcal{J}$ as the monthly mean value of AMOC as in Eqn. \ref{eq:amoc}. Using the adjoint mode of the MITgcm, we compute the sensitivity of the objective function to external forcing: $\pderiv{\mathcal{J}}{\mathcal{F}_k}$, where $\mathcal{F}_k$ is the wind stress and net heat and salt fluxes. These fluxes are computed based on the near surface atmospheric state provided by the ERA-Interim re-analysis fields from 1992-2011 \red{Dee et al, from ECCOv4}. 
	
  We take the objective function to be split into its 20 year mean and monthly anomaly components: 

	\begin{equation}
	  \mathcal{J}(t) = \bar{\mathcal{J}} + \delta\mathcal{J}(t) 
	\end{equation}

  The monthly anomaly in AMOC, $\delta\mathcal{J}(t)$, depends on external forcing such that the AMOC anomaly for a given month can be reconstructed offline via the \red{cross-correlation} integral:
 
	\begin{equation}
	  \delta\mathcal{J}(t) = \sum_k\int_{t_0}^{t}\int_x \int_y\pderiv{\mathcal{J}}{\mathcal{F}_k}(x,y,\tau-t)\delta\mathcal{F}_k(x,y,\tau)\, dxdyd\tau  
	  \label{eq:reconstruct}
	\end{equation}

  where the subscript $k$ refers to each of the momentum and buoyancy fluxes. The integral in time is carried out over the interval $[t_0, t]$ so that this range denotes the ``memory'' of the objective function. In reality each of these integrals are a summation over each dimension, where time is discretized into monthly values. The reconstruction for AMOC at 34$^{\circ}$ is shown in Fig. \ref{fig:fullReconstruction}, where the memory time interval is roughly \red{4 years}.
 
  \red{Proof for why this works?}  

   \begin{figure}
    \centering
    \includegraphics[width=\textwidth]{reconstruct.34S/monthly_flux/full_reconstruct}
    \caption{SAMOC reconstructed as in Eqn. \ref{eq:reconstruct} with monthly accumulated sensitivities and atmospheric forcing due to the wind stress and net heat and salt fluxes. ``Memory'' time is \red{4 years}. Monthly values are reconstructed with their associated sensitivities.}
    \label{fig:fullReconstruction}
   \end{figure}

  \subsection{Exploring reconstruction memory \& buoyancy contributions}

  When reconstructing AMOC as in Eqn. \ref{eq:reconstruct} with only December sensitivities ($\pderiv{\mathcal{J}}{\mathcal{F}_k}$), increasing the memory time significantly changes the quality and phase of the reconstruction (see Fig. \ref{fig:phaseChangeAqh}). This phase change arises because the ocean responds to changes in buoyancy flux related variables differently during different months of the year, so the sensitivity profiles are different depending on the month when AMOC is evaluated. See Fig. \ref{fig:bigPlotHflux} for how June and December sensitivities look in different regions of the world. Notably in the ACC, Atlantic, Arctic, and the area around Greenland. See Fig. \ref{fig:bigPlotTauu} for how these sensitivities are almost identical for wind stress. Thus, when performing the heat flux reconstruction with monthly sensitivities (but removing the zeros this time ...), the phase change goes away, only amplitude increases with memory time (see Fig. \ref{fig:hflux}). It makes sense that using December only would cause a phase change in the reconstructed signal, because the $\delta\mathcal{F}_k(x,y,t)$ \& $\pderiv{\mathcal{J}}{\mathcal{F}_k}$ fields may introduce interference for summer vs winter months (for example).

  
   \begin{figure}
    \centering
    \includegraphics[width=.7\textwidth]{reconstruct.34S/aqh_reconstruct}
    \caption{AMOC variability at 34S with contribution from reconstructing only with air humidity contribution. Here only a December sensitivity is used, note the phase change that is caused.} 
    \label{fig:phaseChangeAqh}
   \end{figure}

   \begin{figure}
    \centering
    \includegraphics[width=\textwidth]{region-mean/adjMean_hflux}
    \caption{Spatial mean of sensitivity for various regions around the globe. Note how the area around Greenland, Atlantic, Arctic, and ACC have different responses for June and December cost functions. There is enough of a difference that the global sensitivity has a different sign for the two months.}
    \label{fig:bigPlotHflux}
   \end{figure}

   \begin{figure}
    \centering
    \includegraphics[width=\textwidth]{region-mean/adjMean_tauu}
    \caption{Same as the last figure but note how the sensitivity profile for each region looks identical in the eyeball norm, merely translated a few months ahead.}
    \label{fig:bigPlotTauu}
   \end{figure}

   \begin{figure}
    \centering
    \includegraphics[width=.7\textwidth]{reconstruct.34S/monthly_flux/hflux_reconstruct}
    \caption{SAMOC variability at 34S with reconstruction contribution from heat flux. Here the monthly variability is reconstructed using a sensitivity based on that month's objective function evaluation. Note how no phase change arises. }
    \label{fig:hflux}
   \end{figure}

  So this seems to imply that it is actually important to compute monthly sensitivities for an accurate reconstruction. However, the reconstruction from wind stress essentially ``nails it''. See Fig. \ref{fig:windReconstruct}. The error $\delta\mathcal{J}(t)_{full} - \delta\mathcal{J}(t)_{wind}$ is quite noisy and has no notable peaks in its frequency spectrum other than at $1 \& 2 yr^{-1}$, see Figs. \ref{fig:windRecError} \& \ref{fig:freqWindRecError}. Additionally, when considering the two norm of the reconstruction error from the forward model, the quality only gets worse when buoyancy fluxes are added (Fig. \ref{fig:reconstructError}). This makes me curious 'what is missing' from the wind only reconstruction. Is this the only important mechanism that we can detect at 34S? Is this the only latitude where this is true or is this the same for all southern latitudes?  

  Compare the reconstruction at 34S to that at 26N. There we can see a notable pattern in the error $\delta\mathcal{J}(t)_{full} - \delta\mathcal{J}(t)_{wind}$, Fig \ref{fig:windRecError26N}, and there is a strong peak at 1/10 $yr^{-1}$ in the frequency spectrum (Fig. \ref{fig:freqWindRecError26N}). I can't say anything about adding heat/salt flux here to make up for this component though because I only have December evaluated objective functions completed. However, the shape looks similar to that in Pillar \cite{pillar}, although the amplitude is smaller. 

   \begin{figure}
    \centering
    \includegraphics[width=.7\textwidth]{reconstruct.34S/monthly_flux/wind_reconstruct}
    \caption{Reconstruction only using wind stress}
    \label{fig:windReconstruct}
   \end{figure}

   \begin{figure}
    \centering
    \includegraphics[width=.7\textwidth]{reconstruct.34S/monthly_flux/fwd_minus_wind}
    \caption{34S: $\delta\mathcal{J}(t)_{full} - \delta\mathcal{J}(t)_{wind}$. Reconstruction is with only December sensitivity so that no gaps exist and we can look at the frequency spectrum in the next plot. For low pass filter, all frequencies $\geq2$ cycles per year were removed. 1 cycle per year was also set to 0. }
    \label{fig:windRecError}
   \end{figure}

   \begin{figure}
    \centering
    \includegraphics[width=.7\textwidth]{reconstruct.34S/monthly_flux/freq_fwd_minus_wind}
    \caption{34S: Frequency domain of $\delta\mathcal{J}(t)_{full} - \delta\mathcal{J}(t)_{wind}$. Reconstruction only with December sens. There are no notable peaks other than the seasonal cycle peaks.}
    \label{fig:freqWindRecError}
   \end{figure}

   \begin{figure}
    \centering
    \includegraphics[width=.7\textwidth]{reconstruct.34S/monthly_flux/norm_error}
    \caption{2 Norm of total error incurred for reconstruction as a function of memory time. Reconstruction using monthly sensitivity fields, months without these data are omitted.}
    \label{fig:reconstructError}
   \end{figure}

   \begin{figure}
    \centering
    \includegraphics[width=.7\textwidth]{reconstruct.26N/fwd_minus_wind}
    \caption{26N: $\delta\mathcal{J}(t)_{full} - \delta\mathcal{J}(t)_{wind}$. Same low pass filter as with 34S.}
    \label{fig:windRecError26N}
   \end{figure}

   \begin{figure}
    \centering
    \includegraphics[width=.7\textwidth]{reconstruct.26N/freq_fwd_minus_wind}
    \caption{26N: Frequency domain of $\delta\mathcal{J}(t)_{full} - \delta\mathcal{J}(t)_{wind}$. Reconstruction only with December sens. Here we see a strong peak at 1/10 $yr^{-1}$. Supposedly this should be reconstructed with heat flux as in Pillar \cite{pillar}.}
    \label{fig:freqWindRecError26N}
   \end{figure}

  This leaves a lot of questions: 
  \begin{enumerate}
	\item Is wind stress the only important component in AMOC variability that is detectable over a 20 yr time frame? and with ECCO
	\item Is this true for the Southern hemisphere or only at 34S? \red{This I could answer quickly} 
	\item why does the reconstruction get worse with buoyancy fluxes, and why does divergence happen after only 4-5 years? (as opposed to Pillar, 15 yrs)?
	\item Why is there a decadal mode in the north? is this recirculation in the sub tropical gyre? equatorial buffer? 
	\item Why does the hflux contribution only look like a seasonally oscillating cycle while in Pillar they didn't see this andor saw decadal mode? Does this just dominate, and are my plotting scales making this seem like it is the case? 
 \end{enumerate}

  Considering signs: I am using the following variables for reconstruction with fluxes. 
  \begin{enumerate}
	\item Wind stress: dJdF corresponds to tauutauv, dF corresponds to the diagnostics oceTAUXY or ExfTauxy. Either seems to produce the same result. 
	\item Heat flux: dJ/dF corresponds to exf variable hflux ($>0 => $decrease in Temp.). dF corresponding to oceQnet ($>0 => $ocean heating, increase in Temp), exfQnet ($>0 =>$ ocean cooling, decrease in Temp). The profiles of these look very different in the Arctic... what is the difference? 
	\item Salt flux: dJ/dF corresponds to exf variable sflux ($>0 =>$ increase in salt). dF corresponds to exfEmpmr ($>0 => $increase in salinity). 
  \end{enumerate}

  \subsection{Monthly variability propagation}

  My goal here is to get a measure of the variability in momentum and buoyancy fluxes and determine how this variation effects the AMOC at 34S. Which component sees the most variation? Is the variation in AMOC explained by deviations from climatological mean values for e.g. wind stress? For each flux denoted by subscript $k$:

  \begin{equation}
	\mathcal{F}_k(x,y,t) = \bar{\mathcal{F}_k}(x,y)_{20yr} + \delta\mathcal{F}_k(x,y,t)
  \end{equation}

  \begin{equation}
	\delta\mathcal{F}_k(x,y,t) = \delta\mathcal{F}_{km}^{clim}(x,y) + \delta\mathcal{F}_{k}^{anom}(x,y,t)
  \end{equation}

  \begin{equation}
	\delta\mathcal{F}_{km}^{clim}(x,y) = \dfrac{1}{20}\sum_n\delta\mathcal{F}_k(x,y,t=m+n) \qquad n \in [0,12,24,...,240]
  \end{equation}

  So what I tried is to reconstruct $\mathcal{J}(t)'$ with:

  \begin{equation}
	\delta\mathcal{F}_k(x,y,t) = \delta\mathcal{F}_{km}^{clim}(x,y) \pm \sigma_{km}(x,y)	
  \end{equation}

  where $\sigma_{km}(x,y)$ is the standard deviation of $\delta\mathcal{F}_k(x,y,t)$, my measure of deviation from climatological mean. 

  \red{Need to try this again when taking std of F\_k not the anomaly from F\_clim}

  \subsection{Forward Perturbation Experiments} 

  In order to test the linearity assumption in the reconstruction I perturbed the heat flux in a forward model similarly to \cite{czeschel}. Here I added a perturbation of $10 W/m^2$ to the region around Greenland (the area with the longest lasting sensitivities). 
  \begin{figure}
	\centering
	\includegraphics[width=.7\textwidth]{hf-perturb/oceqnet__m10}
	\caption{Perturbation in the variable oceQnet is $10 W/m^2$ over the box $45-80^{\circ}N$, inside the Atlantic. Note that since this is just oceQnet the perturbation doesn't show up where there is ice.}
	\label{fig:oceqnet_m10}
  \end{figure}

  \begin{figure}
	\centering
	\includegraphics[width=.7\textwidth]{hf-perturb/amocDiff_}
	\caption{Difference in the AMOC between perturbed and reference experiments. Different responses are shown for 34S and 26N. Both ``hot'' and ``cold'' experiments are shown. \red{Unsure of why the reconstruction turns to an oscillating square wave...}}
	\label{fig:amocDiff}
  \end{figure}

  
\begin{thebibliography}{3}

  \bibitem{meinenSamoc}
  Meinen, et al. MOC Variability at 34.5S...
  
  \bibitem{forgetECCOv4}
  Forget, Gael, et al. \textit{ECCO version 4: an integrated framework for non-linear inverse modeling and global ocean state estimation.} Geoscientific Model Development 8 (2015): 3071-3104.
  
  \bibitem{wunschLinear}
  Carl Wunsch, ``Covariances and linear predictability of the Atlantic Ocean'', Deep Sea Research Part II: Topical Studies in Oceanography, Volume 85, January 2013, Pages 228-243, ISSN 0967-0645, http://dx.doi.org/10.1016/j.dsr2.2012.07.015.
  
  \bibitem{wunschAndHeimbach_AMOC}
  Wunsch, Carl, and Patrick Heimbach. ``Two decades of the Atlantic meridional overturning circulation: Anatomy, variations, extremes, prediction, and overcoming its limitations." Journal of Climate 26.18 (2013): 7167-7186.
  
  \bibitem{lozier}
  Lozier, ``Overturning assumptions.'' 

  \bibitem{mcCarthy}
  McCarthy, G. D., et al. ``Measuring the Atlantic meridional overturning circulation at 26 N." Progress in Oceanography 130 (2015): 91-111.

  \bibitem{heimbach_timescales}
  Heimbach, P., et al. ``Timescales and regions of sensitivity ...''

  \bibitem{pillar}
  Pillar et al ``Dynamical Attribution of Recent Variability in Atlantic Overturning''

  \bibitem{czeschel}
  Czeschel et al: oscillatory sensitivity 

  \bibitem{redi}
  Redi, 1982

  \bibitem{ggl90}
  Gaspar et al., 1990

  \bibitem{gm}
  Gent and McWilliams, 1990

  \bibitem{marotzke}
  Marotzke's paper on sensitivity for physically meaningful objective function

  \bibitem{yaeger2004}
  Yaeger's bulk formula paper

\end{thebibliography}

\end{document}
