\documentclass[a4paper,11pt]{article}
\usepackage[utf8]{inputenc}
\usepackage{geometry,amsmath,color,graphicx}
\geometry{margin=1in}


%opening
\title{\vspace{-10ex}AMOC at 34S}
\author{Tim Smith}
\date{\vspace{-3ex}}

%% --- New commands
\newcommand{\pderiv}[3][]{% \pderiv[<order>]{<func>}{<var>} 
  \ensuremath{\frac{\partial^{#1} {#2}}{\partial {#3}^{#1}}}}
  
\newcommand{\red}[1]{\textcolor{red}{#1}}
\newcommand{\degSym}{$^{\circ}$}

%% --- Figures
\graphicspath{ {../../samoc/figures/} }
\DeclareGraphicsExtensions{.png,.pdf}

\begin{document}

\maketitle

\begin{abstract}
  We are interested in the variability of AMOC at 34$^{\circ}$S over time scales ranging from a few months to ~10-20 years. This location is of particular interest because there are two pilot boundary arrays located here (at 34.5$^{\circ}$): on the Eastern boundary two sensors as part of the GoodHope program (Feb2008-Dec2010) and the Southwest Atlantic MOC array (deployed March 2009) on the Western boundary \cite{meinenSamoc}. These systems are made up of Inverted Echo Sounders (IES), with a pressure sensor (PIES), and in some places a current meter (CPIES). While these arrays were not put in place to monitor AMOC, we use the data provided to compare computer model calculations of the AMOC. Here we use the MITgcm parameters fit to observations as in ECCO v4 r2 (much more here) \cite{forgetEccov4}. Defining the AMOC as an objective function and using the adjoint mode of this code allows us to compute sensitivities of AMOC to various external forcings (wind stress, buoyancy fluxes, ...). The purpose here is to find the causal mechanisms of variability and the timescales on which they act. While a 20 year time frame is not nearly enough to establish statistically meaningful trends in AMOC's behavior, it is certainly true that continuing or reestablishing observation systems will give a clearer picture of AMOC's low frequency variability and trends. 
\end{abstract}

\section{Bullets}
Some interesting finds after computing 20 year adjoint sensitivity fields of AMOC to external forcing fields. 
\begin{enumerate}
	\item After some readjustment time, the sensitivity fields seem to be identical between a december and July cost function. How to compare this more rigorously? 
	\item The RMS (or really L2 norm) and mean behavior of sensitivities looks relatively identical 
	\item Comparing the sensitivities to the exact same setup computed at 25N shows that at 25 north 
\end{enumerate}

\section{Methods}
\label{methods}
  
  \subsection{AMOC definition}
  
    The domain of interest is the zonal cross section of the Atlantic ocean at 34$^{\circ}$S, from the Western coast in the Aghulas Basin to the Cape of Good Hope Basin in the East. In cartesian coordinates, we denote this as: $x \in [x_W,x_E], y = y(\phi=34^{\circ}S), z \in [-H,\eta(y,t)]$ (spherical coordinates are used in the computation, however).  We define the meridional overturning stream function as follows: 
    
    \begin{equation}
      \pderiv[]{\psi_{MOC}}{y}(y,z,t) = \int_{x_W}^{x_E}w(x,y,z,t)\,dx \qquad\text{;}\qquad \pderiv[]{\psi_{MOC}}{z}(y,z,t) = -\int_{x_W}^{x_E}v(x,y,z,t)\,dx .
     \label{eq:mocStf}
    \end{equation}

   Assuming no normal flow at $z = -H$, the stream function is given by:
   \begin{equation}
    \psi_{MOC}(y,z,t) = -\int_{-H}^{z}\int_{x_W}^{x_E}v(x,y,z,t)\,dx\,dz .
    \label{eq:mocStf2}
   \end{equation}

   Note that with this definition, volume transport moving northward is positive. AMOC is defined as the maximum value of this stream function: 
   
   \begin{equation}
    AMOC(y,z_{max},t) = \max_{-H < z < \eta(y,t)}{\psi_{MOC}(y,z,t)} ,
    \label{eq:amoc}
   \end{equation}

   where it is important to note that this value is a function of time and latitude (here denoted as $y$). Here we use monthly mean values for meridional velocity ($\bar{v}$), so that AMOC as a monthly mean quantity. Using results from ECCOv4r2 \cite{forgetEccov4}, the overturning stream function along with its mean and standard deviation over the period 1992-2011 is shown in Fig. \ref{fig:mocStfStats}. At this location and over this time frame, AMOC has a mean value of 14.25 Sv with a standard deviation of 3.14 Sv, where the maximizing depth has a mean and standard deviation 1422m +/- 56m (median depth 1461.4m). Note that the vertical grid spacing at this depth is approximately 100m. \red{Mention that AMOC is normally distributed, cite \cite{wunschAndHeimbach_AMOC}? give measure of normality?}. These are shown in Fig. \ref{fig:samocStats}. Compared to calculations by Meinen et al \cite{meinenSamoc}, these values are slightly lower in the mean, and show far less variation on monthly time periods: they show, in some months, variation up to 30 Sv. I'll discuss these details in section \ref{obsSAMOC}. 

   
   \begin{figure}
    \centering
    \includegraphics[width=\textwidth]{mocStfStats}
    \caption{Overturning stream function at 34\degSym S computed using monthly mean velocity fields from ECCOv4r2.}
    \label{fig:mocStfStats}
   \end{figure}
   
   \begin{figure}
    \centering
    \includegraphics[width=\textwidth]{samocStats}
    \caption{(Top) AMOC computed at 34\degSym S, using monthly mean velocity fields from ECCOv4r2. (Bottom) The maximizing depth at which AMOC is computed during that month. The mean value and +/-1$\sigma$ values are given by the solid gray and dashed gray lines, respectively.}
    \label{fig:samocStats}
   \end{figure}

   
   \subsection{Bolus transport considerations}
   \label{bolus}
   
   The meridional overturning stream function due to bolus transport is shown in Fig. \ref{fig:bolusMocStfStats}. Interestingly there is only significant variation in the mixed layer, and very little beyond ~200m. I originally would have expected more variation since this is the ``eddy transport velocity''. Do we not see variation because I'm using monthly mean values? 
   
   Including the bolus transport increases the 20 year mean AMOC value to 16 Sv, and raises the mean depth to ~1397m (median 1356.9m) (as opposed to 14.25 Sv, 1460m respectively). The standard deviations remain essentially unchanged at 3.15 Sv and 62m. 
   
   \begin{figure}
    \centering
    \includegraphics[width=\textwidth]{bolusMocStfStats}
    \caption{Overturning stream function at 34\degSym S due to bolus transport. Values computed using monthly mean bolus velocity fields from ECCOv4r2.}
    \label{fig:bolusMocStfStats}
   \end{figure}
   
   \begin{figure}
    \centering
    \includegraphics[width=\textwidth]{res_samocStats}
    \caption{(Top) AMOC computed at 34\degSym S, using monthly mean velocity \& bolus velocity fields ($\bar{\mathbf{v}}+\tilde{\mathbf{v}}$) from ECCOv4r2. (Bottom) The maximizing depth at which AMOC is computed during that month. The mean value and +/-1$\sigma$ values are given by the solid gray and dashed gray lines, respectively. Including the bolus transport increases the 20 year mean AMOC value to 16 Sv, and raises the depth to ~1360m (as opposed to 14.25 Sv, 1460m respectively).}
    \label{fig:res_samocStats}
   \end{figure}

   \subsection{Observers SAMOC}
    \label{obsSAMOC}
    
    Following e.g. Meinen et al \cite{meinenSamoc} and McCarthy \cite{mcCarthy}, we construct the observers AMOC at 34\degSym S with the following components: 
    
    \begin{enumerate}
     \item Ekman transport: Computed using monthly mean of 6 hour ERA-Interim re-analysis fields (Dee et al. 2011), interpolated to ocean grid at surface over 1992-2011. The volumetric transport at latitude $y$ is: %
     
     \begin{equation}
      T_{ek}(y,t) = -\int_W^E\frac{\tau_x}{f\rho} \, dx
      \label{eq:ekTransp}
     \end{equation}
     
     where $f$ is the Coriolis paramter at 34\degSym S, $\rho = \rho_0 = 1029 kg/m^3$. 
     
     \item Geostrophic transport: Computed based on the geopotential height anomaly:
     
     \begin{equation}
      T_{geo}(y,z,t) = \int_W^E(v-v_r)dx = \dfrac{1}{f}\int_W^E \dfrac{dp}{\rho} = - \dfrac{1}{f} \int g\,dz = \dfrac{1}{f} \int d\phi= \dfrac{1}{f} (\phi'_E(z) - \phi'_W(z))
      \label{eq:geoTransp}
     \end{equation}
     
     where $\phi'$ is the geopotential surface anomaly which is made up from the surface and hydrostatic components:
     
     \begin{equation}
      \phi'_s \simeq \eta b_0 = \eta g \qquad\text{;}\qquad \phi'_{hyd} = \int_z^{z_{surf}} (b-b_0)dz .
      \label{eq:phiEqns}
     \end{equation}
    
    \item Reference transport: since the geostrophic component is computed relative to a certain level of flow, this is required. Here I use bottom pressure to compute flow at the ocean floor for the reference flow: 
    
    \begin{equation}
     T_{ref}(y,t) = \dfrac{1}{f} \int d\phi= \dfrac{1}{f} (\phi'_E(z) - \phi'_W(z))
     \label{eq:refTransp}
    \end{equation}

    where here bottom pressure anomaly divided by density is used for $\phi'$. 
 
    \end{enumerate}
    
    Currently the geostrophic component is not correct, giving an overturning streamfunction profile as shown in Fig. \ref{fig:geoOvStf}
    
    \begin{figure}
    \centering
    \includegraphics[width=\textwidth]{geoOvStf}
    \caption{Overturning streamfunction calculated with transport defined in Eqn. \ref{eq:geoTransp}}
    \label{fig:geoOvStf}
   \end{figure}

   
  \subsection{Linear Sensitivity}
  \label{linearSensitivity}
  
  We define an objective function $\mathcal{J}$ as the monthly mean value of AMOC as in Eqn. \ref{eq:amoc}. Using the adjoint mode of the MITgcm, we then compute the sensitivity of the objective function to model inputs. In this case, the inputs consist of external forces: wind stress, heat, and freshwater fluxes. 
  

\begin{thebibliography}{3}

  \bibitem{meinenSamoc}
  Meinen, et al. MOC Variability at 34.5S...
  
  \bibitem{forgetEccov4}
  Forget, Gael, et al. \textit{ECCO version 4: an integrated framework for non-linear inverse modeling and global ocean state estimation.} Geoscientific Model Development 8 (2015): 3071-3104.
  
  \bibitem{wunschLinear}
  Carl Wunsch, ``Covariances and linear predictability of the Atlantic Ocean'', Deep Sea Research Part II: Topical Studies in Oceanography, Volume 85, January 2013, Pages 228-243, ISSN 0967-0645, http://dx.doi.org/10.1016/j.dsr2.2012.07.015.
  
  \bibitem{wunschAndHeimbach_AMOC}
  Wunsch, Carl, and Patrick Heimbach. ``Two decades of the Atlantic meridional overturning circulation: Anatomy, variations, extremes, prediction, and overcoming its limitations." Journal of Climate 26.18 (2013): 7167-7186.
  
  \bibitem{mcCarthy}
  McCarthy, G. D., et al. ``Measuring the Atlantic meridional overturning circulation at 26 N." Progress in Oceanography 130 (2015): 91-111.

\end{thebibliography}

\end{document}
