\documentclass[a4paper,11pt]{article}
\usepackage[utf8]{inputenc}
\usepackage{geometry,amsmath,color,graphicx}
\geometry{margin=1in}


%opening
\title{\vspace{-10ex}Offline SAMOC Reconstruction}
\author{Tim Smith}
\date{\vspace{-3ex}}

%% --- New commands
\newcommand{\pderiv}[3][]{% \pderiv[<order>]{<func>}{<var>} 
  \ensuremath{\frac{\partial^{#1} {#2}}{\partial {#3}^{#1}}}}
  
\newcommand{\red}[1]{\textcolor{red}{#1}}
\newcommand{\degSym}{$^{\circ}$}

%% --- Figures
\graphicspath{ {../../samoc/figures/} }
\DeclareGraphicsExtensions{.png,.pdf}

\begin{document}

\maketitle
  \section{Linear Sensitivity}
  \label{linearSensitivity}
  
  We define an objective function $\mathcal{J}$ as the monthly mean value of AMOC as in Eqn. \ref{eq:amoc}. Using the adjoint mode of the MITgcm, we compute the sensitivity of the objective function to external forcing: $\pderiv{\mathcal{J}}{\mathcal{F}}$. In this case, $\mathcal{F}$ consists of the near surface air temperature, humidity, wind stress, precipitation, and radiative forcing.  
	
  We take the objective function to be split into its 20 year mean and monthly anomaly components: 
	\begin{equation}
	  \mathcal{J}_i(t) = \bar{\mathcal{J}_i}(t) + \mathcal{J}_i'(t) \qquad i \in [1, 2, ... 12]
	\end{equation}

  Here the subscript $i$ denotes the month in which the cost function is evaluated (January, February, etc). We hypothesize that the monthly anomaly in AMOC, $\mathcal{J}_i'(t)$, is linearly dependent on external forcing. Thus, the AMOC anomaly for a given month, $i$, can be reconstructed offline via the convolution integral:
 
	\begin{equation}
	  \mathcal{J}_i'(t) = \int_{1992}^{t}\int_x \int_y\pderiv{\mathcal{J}_i}{\mathcal{F}}(x,y,\tau-t)\mathcal{F}'(x,y,\tau)\, dxdyd\tau  
	\end{equation}

   
  
  

\end{document} 
